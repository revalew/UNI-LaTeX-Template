\usepackage{geometry}
\geometry{ % sets the paper size and margin widths, allows for customization of page layout
    a4paper,
    % total={210mm,297mm},
    left=3cm,
    right=2.5cm,
    top=2.5cm,
    bottom=2.5cm
}

%---------------------------------------------------------------------------

% adds support for Polish language
\usepackage[]{polski}
\usepackage[polish]{babel}

% font
\usepackage[T1]{fontenc} %font encoding for accent characters - like in polish
% \usepackage[utf8]{inputenc} % input encoding used in the source as utf-8 % In recent LaTeX installations, UTF-8 is the default input encoding, and inputenc package is not required anymore. Therefore, it can be safely removed. However, if legacy files are used, which are not in UTF-8 format, inputenc might be needed to specify the appropriate input encoding for non-ASCII characters.

\usepackage{helvet} % loads the Helvetica font
\renewcommand{\familydefault}{\sfdefault} % sets the default font family to sans-serif, which is the Helvetica font

% \usepackage{fontspec} % set the main font to Arial using the fontspec package
% \setmainfont{Arial}

\usepackage[usenames]{color}

\usepackage{titlesec} % center section titles
\usepackage[]{indentfirst} % indents the first paragraph after a section or subsection heading
\linespread{1.5} % It sets the line spacing to 1.5
% \renewcommand*{\thesubsubsection}{} % removes subsubsection numbering
\usepackage{anyfontsize} % allows using any font size
\usepackage{tabto} % allows setting a tab stop position using the \tab command
%\usepackage[strings]{underscore} % allow using the underscore character (_) in the document
% \usepackage[style=czech]{csquotes}
\usepackage{ragged2e} % provides extra options for text justification
\usepackage{graphicx} % It allows including images in the document using the \includegraphics command
    % interpolate=false => turn off the interpolation
    % trim=4cm 1cm 1cm 1.5cm => L B R T can use pt, cm, in to crop images
    % clip => deletes the part outside of the trim option. otherwise it would just be invisible
% \graphicspath{{img/}}
% \DeclareGraphicsRule{.gif}{eps}{.gif.bb}{`convert #1 eps:-}
% \DeclareGraphicsExtensions{.pdf,.png,.jpg, .jp, .gif} %For pdflatex
\usepackage{dtk-logos} % For BibTeX and other Logos

\usepackage{tikz}

\usepackage{framed} % used to create visually appealing frames or boxes around text, equations, or figures in LaTeX
\usepackage{epstopdf} % allows the user to include EPS images in their LaTeX documents without worrying about compatibility issues with different PDF viewers.

% \usepackage{pdfpages} % enables including external PDF files in the document
\usepackage{standalone} % allows you to compile standalone files that are either part of a larger document or produce graphics to be included in another document.
%---------------------------------------------------------------------------

\usepackage[]{amsfonts} % provides access to additional fonts and symbols from American Mathematical Society
\usepackage[]{amsmath} % provides various math-related commands and environments (matrix)
\usepackage{amssymb} % provides math symbols and commands, such as the greater than or equal to sign (>=)
\usepackage{gensymb} % provides the possibility to use the degree symbol (\degree)
%\usepackage{mathabx} % provide more math symbols and fonts
\usepackage{bm} % allows typesetting bold math symbols using the \bm command
% \usepackage{pgffor} %  provides a loop macro \foreach, which allows easy iteration over lists or ranges of values
% \usepackage[nomessages]{fp} % provides floating point arithmetic operations on numbers, allowing calculations to be made with very high precision

%---------------------------------------------------------------------------

\usepackage{array} % provides enhanced table formatting options
\usepackage{tabularx} % provides a new column type X that automatically adjusts its width so that the table fills the available space
\usepackage{multirow} % allows merging cells vertically in a table using the \multirow command
\usepackage{rotating} % allows rotating text and objects using the \begin{sideways} environment
\usepackage{float} % improves the positioning of floats (tables and figures) in the document
\usepackage{caption} % allows adding captions to figures and tables
\usepackage{subcaption} %  allow creating subfigures and subtables with their own captions
% \usepackage{subfigure} % eps
% \usepackage{subfig}
\usepackage[shortlabels]{enumitem} % provides more options for customizing lists, such as changing the label style or starting from a specific number => a) b) c)
\usepackage{longtable} % allows tables to span multiple pages
\newcolumntype{Y}{>{\centering\arraybackslash}X} % defines a new column type Y that centers its contents horizontally

\usepackage{multicol}
%---------------------------------------------------------------------------

\usepackage{xparse} % provides a powerful syntax for defining new commands
\usepackage{xstring} % provides functions for manipulating strings of text

\usepackage[numbered, mlscaleinline]{matlab-prettifier} % adds code highlighting for Matlab code
\lstset{
    % basicstyle=\fontsize{8}{10}\selectfont\ttfamily
    literate={ą}{{\k{a}}}1
    {Ą}{{\k{A}}}1
    {ę}{{\k{e}}}1
    {Ę}{{\k{E}}}1
    {ó}{{\'o}}1
    {Ó}{{\'O}}1
    {ś}{{\'s}}1
    {Ś}{{\'S}}1
    {ł}{{\l{}}}1
    {Ł}{{\L{}}}1
    {ż}{{\.z}}1
    {Ż}{{\.Z}}1
    {ź}{{\'z}}1
    {Ź}{{\'Z}}1
    {ć}{{\'c}}1
    {Ć}{{\'C}}1
    {ń}{{\'n}}1
    {Ń}{{\'N}}1
}
% \lstset{basicstyle=\fontsize{4}{4.6}\selectfont\ttfamily}

%---------------------------------------------------------------------------
\usepackage{url} % allows formatting URLs using the \url command
\usepackage{hyperref} % creates clickable links for sections, references, and URLs in the document
\hypersetup{ % Here, we are making all the links to be colored and sets their colors respectively as: blue for URL links, black for internal links (e.g., references), black for links to local files and black for citations
    colorlinks,
    citecolor=black,
    filecolor=black,
    linkcolor=black,
    urlcolor=black
}