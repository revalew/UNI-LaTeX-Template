\title{
\centering


% Check for specific options
\IfValueTF{\logoType}{%
    \IfSubStr{\logoType}{pl}{
        \vspace{-5.0cm} % pl logo
        \begin{flushright}
            % [colorinvert] [grayscale]
            \includegraphics[scale = 0.2, trim={0 0 4.6cm 0}, clip]{./../src/logo/pl/polsl_logo.png}\\ % pl logo
        \end{flushright}
        \vspace*{2.7cm} % pl logo
        }{}
    \IfSubStr{\logoType}{en}{%
        \vspace{-4.0cm} % en logo
        \begin{flushright}
            % [colorinvert] [grayscale]
            \includegraphics[width=1.3in]{./../src/logo/en/polsl_logo2_zn_en.pdf}\\ % en logo
        \end{flushright}
        \vspace*{3.3cm} % en logo
    }{}%
}{}%

\IfValueTF{\department}{%
    \IfSubStr{\includeDepartment}{yes}{
        {\fontsize{30}{20}\selectfont\textbf{\department}}\\
        {\fontsize{24}{30}\selectfont\textbf{\subjectName}}\\
        \vspace*{3.2cm}
        {\fontsize{14}{0}\selectfont 
        Ćwiczenie nr \exNumber \\[-1.2em] 
        Temat: \exTopic
        }
        \vspace*{-1cm}
    }{
        \IfSubStr{\includeDepartment}{no}{
            {\fontsize{24}{30}\selectfont\textbf{\subjectName}}\\
            \vspace*{4.2cm}
            {\fontsize{14}{0}\selectfont 
            Ćwiczenie nr \exNumber \\[-1.2em] 
            Temat: \exTopic
            }
        }{}
    }
}


\IfNoValueTF{\thirdAuthorName}{
    \IfNoValueTF{\secondAuthorName}{
        \vspace*{4.2cm} % 1 person in section
        \author{
        % This code starts a new box or a minipage with a width of 100% of the line width. The minipage in LaTeX is similar to a text box in which you can position text similarly to the way you do on a regular page, and surround it with borders (if desired)
        \begin{minipage}{1\linewidth}
            \begin{flushright}
                {\begin{minipage}[t]{.28\linewidth}
                    \groupName \\
                    \authorName
                \end{minipage}}
            \end{flushright}
        \end{minipage}
        }
    }{
        \vspace*{3.8cm} % 2 people in section
        \author{
        % This code starts a new box or a minipage with a width of 100% of the line width. The minipage in LaTeX is similar to a text box in which you can position text similarly to the way you do on a regular page, and surround it with borders (if desired)
        \begin{minipage}{1\linewidth}
            \begin{flushright}
                {\begin{minipage}[t]{.28\linewidth}
                    \groupName \\
                    \authorName \\
                    \secondAuthorName
                \end{minipage}}
            \end{flushright}
        \end{minipage}
        }
    }
}{
    \vspace*{2.9cm} % 3 people in section
    \author{
    % This code starts a new box or a minipage with a width of 100% of the line width. The minipage in LaTeX is similar to a text box in which you can position text similarly to the way you do on a regular page, and surround it with borders (if desired)
    \begin{minipage}{1\linewidth}
        \begin{flushright}
            {\begin{minipage}[t]{.28\linewidth}
                \groupName \\
                \authorName \\
                \secondAuthorName \\
                \thirdAuthorName
            \end{minipage}}
        \end{flushright}
    \end{minipage}
    }
}
}

% \date{\today}
\IfSubStr{\includeDepartment}{yes}{
    \date{\vfill Gliwice \the\year{}\\[\dimexpr\footskip]} % automatically input the current year
}{
    \date{\vfill Gliwice \the\year{}} % automatically input the current year
}

\renewcommand{\contentsname}{Spis treści}

\newcommand{\inputmytitle}{
    \renewcommand*{\figurename}{Rys.} % these commands are defining the specific format for captions of figures and tables respectively, and it's a document-specific choice rather than something that should be applied globally to all documents
    \renewcommand*{\tablename}{Tab.}

    \renewcommand{\lstlistingname}{Kod}% Listing -> Algorithm
    % \renewcommand{\lstlistlistingname}{Spis \lstlistingname ów}% List of Listings -> List of Algorithms

    \renewcommand{\refname}{Bibliografia} % Literatura -> Bibliografia (References)

    \newgeometry{left=1.5cm, right=1.5cm, top=2.5cm, bottom=0.1cm} % Change the bottom margin
    \maketitle % generates the title of the document based on information provided in the preamble of the document, such as title, authors and date
    \restoregeometry % Go back to the original margins

    % %---------------------------------------------------------------------------
%                                DEDICATION

\thispagestyle{empty} % command that removes the page number and other header/footer elements from the current page
\clearpage % ends the current page and causes any remaining floating objects such as figures or tables to be printed, moving them to the next page if needed
\newpage
\
\vfill % inserts vertical blank spaces to fill the page
\hfill % inserts a horizontal blank space that fills the remaining area of a line or row in a table or array producing a flush right effect
\begin{minipage}{.45\linewidth} % This code starts a new box or a minipage with a width of 45% of the line width. The minipage in LaTeX is similar to a text box in which you can position text similarly to the way you do on a regular page, and surround it with borders (if desired)
    \begin{flushright}
        \textit{Dziękuję}
    \end{flushright}
\end{minipage}
\vspace*{3cm}
\thispagestyle{empty}
\clearpage
\pagebreak

%--------------------------------------------------------------------------- % place to toggle the dedication

    \thispagestyle{empty}
    \clearpage
    \tableofcontents

    \newpage
}