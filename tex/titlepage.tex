\title{
\centering

% \vspace{-4.0cm} % en logo
\vspace{-5.0cm} % pl logo

\begin{flushright}
    % [colorinvert] [grayscale]
    % \includegraphics[width=1.3in]{./../src/logo/en/polsl_logo2_zn_en.pdf}\\ % en logo
    \includegraphics[scale = 0.2, trim={0 0 4.6cm 0}, clip]{./../src/logo/pl/polsl_logo.png}\\ % pl logo
\end{flushright}

% \vspace*{3.3cm} % en logo
\vspace*{2.7cm} % pl logo

% {\fontsize{30}{20}\selectfont\textbf{Wydział Automatyki, Elektroniki i~Informatyki}}\\
{\fontsize{24}{30}\selectfont\textbf{Laboratorium \\
Przetwarzania Obrazów Cyfrowych}}\\
\vspace*{4.2cm}

{\fontsize{14}{0}\selectfont 
Ćwiczenie nr 3 \\[-1.2em] 
Temat: Wyznaczanie liczby barw w obrazie (L03).
}
\vspace*{4.2cm} % 1 person in section
% \vspace*{3.8cm} % 2 people in section
% \vspace*{2.9cm} % 3 people in section
}
\author{
% This code starts a new box or a minipage with a width of 100% of the line width. The minipage in LaTeX is similar to a text box in which you can position text similarly to the way you do on a regular page, and surround it with borders (if desired)
\begin{minipage}{1\linewidth}
    \begin{flushright}
        {\begin{minipage}[t]{.28\linewidth}
            AiR gr. 5 \\
            % John Dere \\
            Kisiel Maksymilian
            % \\ Hugh Janus
        \end{minipage}}
    \end{flushright}
\end{minipage}
\vspace*{2.0cm}
}

% \date{\today}
\date{Gliwice 2023}

\renewcommand{\contentsname}{Spis treści}

\newcommand{\inputmytitle}{
    \renewcommand*{\figurename}{Rys.} % these commands are defining the specific format for captions of figures and tables respectively, and it's a document-specific choice rather than something that should be applied globally to all documents
    \renewcommand*{\tablename}{Tab.}
    \newgeometry{left=1.5cm, right=1.5cm, top=2.5cm, bottom=0.1cm} % Change the bottom margin
    \maketitle % generates the title of the document based on information provided in the preamble of the document, such as title, authors and date
    \restoregeometry % Go back to the original margins

    % %---------------------------------------------------------------------------
%                                DEDICATION

\thispagestyle{empty} % command that removes the page number and other header/footer elements from the current page
\clearpage % ends the current page and causes any remaining floating objects such as figures or tables to be printed, moving them to the next page if needed
\newpage
\
\vfill % inserts vertical blank spaces to fill the page
\hfill % inserts a horizontal blank space that fills the remaining area of a line or row in a table or array producing a flush right effect
\begin{minipage}{.45\linewidth} % This code starts a new box or a minipage with a width of 45% of the line width. The minipage in LaTeX is similar to a text box in which you can position text similarly to the way you do on a regular page, and surround it with borders (if desired)
    \begin{flushright}
        \textit{Dziękuję}
    \end{flushright}
\end{minipage}
\vspace*{3cm}
\thispagestyle{empty}
\clearpage
\pagebreak

%--------------------------------------------------------------------------- % place to toggle the dedication

    \thispagestyle{empty}
    \clearpage
    \tableofcontents

    \newpage
}